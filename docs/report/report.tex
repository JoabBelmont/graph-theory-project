\documentclass[12pt, letterpaper]{article}
\usepackage[brazilian]{babel}
\usepackage{geometry}
\geometry{
    a4paper,
    total={170mm,257mm},
    left=20mm,
    top=20mm,
}

\title{\textbf{Projeto Final: \\ Problemas de Grafos}}
\date{}
\author{Antônio Joabe Alves Morais \\ Iarley Natã Lopes Souza}

\begin{document}
    \maketitle

    \section{[Problema 1] Colorindo os vértices do grafo com duas cores}
        \subsection{Descrição da solução}
            Na solução, o primeiro vértice ($v_0$) no grafo é pintado de vermelho, portanto todos os vizinhos deverão ser pintados com a cor contrária: Azul.

            A partir disto, forma-se um efeito cascata: Ao analizarmos um vértice $v$, verificamos seus vizinhos. Se um vizinho $u$ for branco, devemos pintá-lo com a cor contrária de $v$, caso contrário (se $u$ não for branco), comparamos as cores de $v$ e $u$.

            Se $v$ e $u$ possuírem cores diferentes, a execução do algoritmo irá prosseguir normalmente. Se possuírem cores iguais, o algoritmo será interrompido, assim indicando que não é possível bipartir o grafo.

        \subsection{Complexidade}
            A complexidade será dada, de acordo com a notação \emph{Big O}, por:
            $$ O(V + E) $$
            A justificativa se dá pelo fato do algoritmo visitar todos os vértices e arestas, já que precisamos verificar a cor de todos eles, a fim de saber se um grafo é bipartido ou não.

            A notação, portanto, irá indicar que a complexidade de tempo de execução cresce de acordo com a soma desses dois termos, vértices ($V$) e arestas ($E$), de forma diretamente proporcional.

    \section{[Problema 2] Seis graus de Kevin Bacon}
        \subsection{Descrição da solução}
            Neste problema, é necessária apenas a aplicação da \emph{BFS} (Breadth-First Search), já que o algoritmo já conta com o armazenamento das distâncias de todos os vértices dado uma origem, neste caso, Kevin Bacon.

            Portanto, dado um grafo não-direcionado, a \emph{BFS} irá fornecer os números de Bacon que o problema demanda por meio das distâncias armazenadas na execução do algoritmo.
        \subsection{Complexidade}
            Essa solução também exige um tempo de execução:
            $$ O(V + E) $$
            Analizando o algoritmo de busca de forma agregada, temos que o tempo de inicialização onde se visita os vértices se dá por $O(V)$, assim como o tempo gasto com a fila.

            Além disso, ao percorrer as adjacências (arestas) gasta-se $O(E)$.

            Portanto, conforme já difundido, a análise de complexidade de um algorimo \emph{BFS} é $O(V + E)$.
    \section{[Problema 3] Vias de mão dupla}
        \subsection{Descrição da solução}
        \subsection{Complexidade}
\end{document}